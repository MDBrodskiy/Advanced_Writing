%%%%%%%%%%%%%%%%%%%%%%%%%%%%%%%%%%%%%%%%%%%%%%%%%%%%%%%%%%%%%%%%%%%%%%%%%%%%%%%%%%%%%%%%%%%%%%%%%%%%%%%%%%%%%%%%%%%%%%%%%%%%%%%%%%%%%%%%%%%%%%%%%%%%%%%%%%%%%%%%%%%
% Written By Michael Brodskiy
% Class: Advanced Writing in the Disciplines
% Professor: L. Nardone
%%%%%%%%%%%%%%%%%%%%%%%%%%%%%%%%%%%%%%%%%%%%%%%%%%%%%%%%%%%%%%%%%%%%%%%%%%%%%%%%%%%%%%%%%%%%%%%%%%%%%%%%%%%%%%%%%%%%%%%%%%%%%%%%%%%%%%%%%%%%%%%%%%%%%%%%%%%%%%%%%%%

\documentclass[12pt]{article} 
\usepackage{alphalph}
\usepackage[utf8]{inputenc}
\usepackage[russian,english]{babel}
\usepackage{titling}
\usepackage{amsmath}
\usepackage{graphicx}
\usepackage{enumitem}
\usepackage{amssymb}
\usepackage[super]{nth}
\usepackage{everysel}
\usepackage{ragged2e}
\usepackage{geometry}
\usepackage{multicol}
\usepackage{fancyhdr}
\usepackage{cancel}
\usepackage{siunitx}
\usepackage{setspace}
\usepackage[table]{xcolor}

\doublespacing

\geometry{top=1.0in,bottom=1.0in,left=1.0in,right=1.0in}
\newcommand{\subtitle}[1]{%
  \posttitle{%
    \par\end{center}
    \begin{center}\large#1\end{center}
    \vskip0.5em}%

}
\usepackage{hyperref}
\hypersetup{
colorlinks=true,
linkcolor=blue,
filecolor=magenta,      
urlcolor=blue,
citecolor=blue,
}

\usepackage{tcolorbox}

\pagestyle{fancy}

\lfoot[\vspace{-15pt} \hline]{\vspace{-15pt} \hline}
\rfoot[\vspace{-15pt} \hline]{\vspace{-15pt} \hline}
\cfoot[\thepage]{\thepage}
\chead[\textsc{Advanced Writing in the Disciplines}]{\textsc{Advanced Writing in the Disciplines}}
\lhead[\textsc{Project 1}]{\textsc{Project 1}}
\rhead[\textsc{ENGW3302}]{\textsc{ENGW3302}}



\begin{document}

\begin{center}

  \textbf{\underline{Final Reflection}}

\end{center}

\begin{justify}

  \hspace{.5in} Admittedly, I came into this course quite cynical; that is, tackling a writing course on top of my co-op (which already involves intensive document fabrication and review) and capstone course (which involves the creation of a project charter and working simulation) seemed unreasonable. Furthermore, frequent travel, during which a stable internet connection is not feasible, only exacerbated this concern. As I began delving into the assigned projects, I found them to be not only reasonable and useful, but all the more pleasurable, as it allowed me to reflect on both my thoughts and current field of study. As such, I used the assigned projects as an outlet for my creativity and philosophy — I let my thoughts run wild, and would then reshape them by referring to the plentiful peer reviews. As such, I was able to refine my existing arguments for improved future use.

  \hspace{.5in} When I say ``existing arguments,'' I am primarily referring to my staunch support of the Free Software philosophy, and, consequently, the project I found most influential: producing an article (Project 3). For roughly five years now, I have not only adhered to (as closely as possible in a modern educational environment), but also supported the ideals promoted by the Free Software Foundation (FSF). Despite numerous publications, extensive rhetorical experience, and even speaking at Libreplanet, the FSF's annual conference, my journey through this philosophy continues to focus on improving my ``elevator pitch'' arguments. In doing so, I am able to become a more robust proselytizer. Accordingly, by writing my crude thoughts and being afforded the opportunity of peer reviews, I was able to develop this.

  \hspace{.5in} In addition to improved argumentation skills, the rapid and prolific production of written works allowed me to uncover trends in my linguistic weaknesses. For example, it seemed that producing overly wordy writing occurred quite frequently across my projects. For example, a peer review from Project 1 states ``I wasn't entirely clear [on] what the issue at hand was here by the end,'' which parallels Project 2 ``[Y]ou could be a little more to the point'' and Project 3 ``[Y]ou jump around a bit between ideas.'' Given that this was the most prominent similarity between peer comments on my publications, I will try to improve by producing more concise and readable works. My progress is already visible across the three projects, as, in my Project 3 Author's Note, I immediately state ``Beginning the revision process, I found it best to begin by simply re-reading the entire passage. As noted by a peer, some areas appeared to be wordy . . .'' This is in contrast to the initial projects, where I would only jump from section to section, which proved to be detrimental to overall cohesion. Additionally, it did seem that, overall, I have a robust sense of professional writing. For example, a peer review from Project 3 states ``[T]his is some very clear and concise writing,'' which is in agreement with Project 2's ``From the first look, I [the peer reviewer] wouldn't be able to tell the difference between this and some of the journal articles I've read.''

  \hspace{.5in} While the article allowed me to focus on the ``raw'' output of my thoughts, the research project allowed me to more professionally analyze Free Software issues. In referencing research and conference proceedings, I was able to uncover hard evidence of trends, whereas I usually relied on provoking critical thinking. For example, when proselytizing, I often ask another individual ``would you rather many thousands of people review and contribute to a project of their own free will and personal interest, or ten or so people who are paid to do it as a career?'' in order to evoke a thoughtful response. In addition to this, I can now add actual evidence, like the case studies from Project 2, which indicate improved social and technical skills, in addition to greater inclusivity from individuals who used free software. As such, this only furthers my goal of being a more convincing speaker regarding this philosophy. I will continue to use these developed skills to promote a greater sense of community among both existing and (possibly unknowing) future supporters, which, ultimately, may solve the problem presented in this project.

  \hspace{.5in} Although the article and research project allowed me to develop myself rhetorically, I was able to connect on a more personal level with the disciplinary introduction (Project 1). I found the project to be quite unique in that, rather than producing a technical document related to my field, it was more reflective, and allowed me to uncover more about the field and, in doing so, myself. Albeit enjoyable for me, I do think I focused a bit too much on my personal experiences working in the field of physics, and, as such, I would focus less on this and more on including greater detail related to physicists in general.

  \hspace{.5in} Overall, the class allowed me to develop existing argumentation skills in a field I deeply care about — I will be sure to apply these improved skills in the future. 

\end{justify} 

\end{document}

