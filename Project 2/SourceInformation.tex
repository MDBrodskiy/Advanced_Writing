%%%%%%%%%%%%%%%%%%%%%%%%%%%%%%%%%%%%%%%%%%%%%%%%%%%%%%%%%%%%%%%%%%%%%%%%%%%%%%%%%%%%%%%%%%%%%%%%%%%%%%%%%%%%%%%%%%%%%%%%%%%%%%%%%%%%%%%%%%%%%%%%%%%%%%%%%%%%%%%%%%%
% Written By Michael Brodskiy
% Class: Advanced Writing in the Disciplines
% Professor: L. Nardone
%%%%%%%%%%%%%%%%%%%%%%%%%%%%%%%%%%%%%%%%%%%%%%%%%%%%%%%%%%%%%%%%%%%%%%%%%%%%%%%%%%%%%%%%%%%%%%%%%%%%%%%%%%%%%%%%%%%%%%%%%%%%%%%%%%%%%%%%%%%%%%%%%%%%%%%%%%%%%%%%%%%

\include{Includes.tex}

\title{Adapted Resource Matrix}
\date{\today}
\author{Michael Brodskiy\\ \small Professor: L. Nardone}

\begin{document}

\maketitle

\begin{enumerate}

  \item ``FLOSS in Software Engineering Education: Supporting the Instructor in the Quest for Providing Real Experience for Students''

    \begin{itemize}

      \item Authors: Fernanda Gomes Silva, Moara Sousa Brito, Jenifer Vieira Toledo Tavares, Christina von Flach G. Chavez

      \item Journal: Association for Computing Machinery Digital Library

      \item Published: 23 September 2019

      \item doi: 10.1145/3350768.3353815

      \item Journal is reputable source, provided by Northeastern University Library access. Contributors are students from the Federal University of Bahia, Salvador, Brazil

        \begin{itemize}

          \item Fernanda Gomes Silva has other, similar publications regarding the Open Source Model (OSM) and implementations in education.

          \item Moara Sousa Brito has one other recent publication, also related to Free/Libre and Open Source Software (FLOSS).

          \item This is Jenifer Vieira Tavares's only publication

          \item Christina von Flach G. Chavez has one other publication related developer reasoning on module cohesion (loosely related, as both concern software-related problems)

        \end{itemize}

      \item This publication considers the benefits of implementing FLOSS programs and methodologies in software engineering education. The researchers performed a classroom case study by assisting a professor in setting up a class via FLOSS applications, and then having students use a handful of pre-selected FLOSS applications for coursework. 

      \item The authors find that student experience with FLOSS programs enhanced their ability to handle personal projects, as well as code provided from others. This supported students standing in the job market. Furthermore, students became more proactive and communicative. The instructor reported greater interactions between students, as well as the group being dynamic and enthusiastic.

      \item This source seems to agree with source 2 in that FLOSS is greatly beneficial when it comes to educational environments, and, furthermore, that FLOSS spurs greater interest in learning. This source has little relation to source 3, as this source focuses more on the actual use of FLOSS (in an educational environment), while source 3 discusses difficulties with supporting FLOSS projects. As source 4 is an exposition piece on an existing FLOSS project, source 1 would seem to agree with the how beneficial this program would be, especially within an educational environment. The arguments laid out in source 5 are confirmed in the case study provided in source 1. Source 6 provides a definition, which is further expanded upon in source 1.

      \item Implementing this source will assist me in arguing for the implementation of FLOSS projects in educational establishments, which, in and of itself, is a solution to the freedom-denying program epidemic.

    \end{itemize}

  \item ``How FLOSS Participation Supports Lifelong Learning and Working: Apprenticeship Across Time and Spatialities''

    \begin{itemize}

      \item Author: Aditya Johri

      \item Journal: Association for Computing Machinery Digital Library

      \item Published: August 2018

      \item doi: 10.1145/3233391.3233541

      \item Journal is reputable source, provided by Northeastern University Library access. Contributor is a well-regarded individual, and has attended George Mason University, Virginia Polytechnic Institute, and Stanford University. The author has over 64 published papers, mostly exploring solutions to existing problems in education.

      \item This publication analyzes two case studies pertaining to participatory learning in FLOSS. The two cases involve tracing professional performance of two students from high school, into the workforce, and beyond.

      \item The author concludes that FLOSS supports lifelong learning, as it is an ecosystem that allows participants to learn both technical and nontechnical skills with use. Furthermore, the author argues that FLOSS is an ideal, networked form of learning.

      \item This source would agree with the findings presented in source 1, and would support the implementation of FLOSS exemplified in source 4. This source confirms arguments presented in source 5, and implements the same, official definition given in source 6. Source 3 would neither agree nor disagree, but rather presents difficulties in supporting such projects.

      \item Similar to source 1, these findings can allow me to further confirm that FLOSS projects and education go hand in hand.

    \end{itemize}

  \item ``The Labor of Maintaining and Scaling Free and Open-Source Software Projects''

    \begin{itemize}

      \item Authors: R. Stuart Geiger, Dorothy Howard, Lilly Irani

      \item Journal: Association for Computing Machinery Digital Library

      \item Published: 22 April 2021

      \item doi: 10.1145/3449249

      \item Journal is reputable source, provided by Northeastern University Library access. Authors are from the University of California, San Diego

        \begin{itemize}

          \item R. Stuart Geiger has over 20 publications, most of which pertain to cooperative work in the digital realm.

          \item Dorothy Howard has 2 publications, and, though the other is not related to FLOSS programs, it does pertain to project management.

          \item Lilly Irani has 35 publications, most of which concern computing, more specifically computing for the user, by the user.

        \end{itemize}

      \item Through an interview-based approach, the contributors investigate the difficulties of maintaining FLOSS projects. Given that a major part of FLOSS is the community, project maintainers not only fix bugs, deploy security patches, and update dependencies, but also interact with the community. Thus, the contributors decided to learn more about how this plays a role in maintaining and scaling such projects.

      \item Though the authors agree that FLOSS projects benefit all, in general, they conclude that scaling FLOSS projects becomes very difficult. This is not only due to the need for labor, but also the emotional struggle that comes with being attached to one's project, as well as possible roadblocks and/or interference from governance or other companies.

      \item Though the article uses the definition laid out in source 6, it is more or less independent from the other sources in that it focuses on a (possibly) persistent problem with FLOSS.

      \item The use of this article would allow me to shed some light not only on the user perspective, but also the contributor perspective of FLOSS projects.

    \end{itemize}

  \item ``LIBRE-ary, an open-source, distributed digital archiving system''

    \begin{itemize}

      \item Author: Ben Glick, Jens Mache

      \item Journal: Association for Computing Machinery Digital Library

      \item Published: 1 October 2019

      \item doi: 10.5555/3381540.3381543

      \item Journal is reputable source, provided by Northeastern University Library access. Both contributors attend Lewis and Clark University

        \begin{itemize}

          \item Ben Glick has one other publication, also related to education in computer science.

          \item Jens Mache has over 75 publications, and, though not all directly related to FLOSS, most have to do with contributions to software projects.

        \end{itemize}

      \item This source presents a management solution to the archiving problem of ``digital clutter'' created by the contributors. This solution is FLOSS.

      \item The solution is a provided as a FLOSS program, created by the contributors.

      \item This source is unique in that it directly exemplifies an existing program, rather than performing a meta-analysis or case study. It would, however, agree that the use of FLOSS is greatly beneficial in education.

      \item This source will allow me to provide an example of a FLOSS program.

    \end{itemize}

  \item ``Why Schools Should Exclusively Use Free Software''

    \begin{itemize}

      \item Author: Dr. Richard M. Stallman

      \item Journal: Free Software Foundation Archive

      \item Published: No date provided

      \item doi: \href{https://www.gnu.org/education/edu-schools.html}{https://www.gnu.org/education/edu-schools.html}

      \item This source considers the importance of Free Software program usage in classrooms.

      \item The author argues that schools and educational establishments \underline{must} use exclusively Free Software alternatives.

      \item This source would agree with the findings presented in the first two sources, and with the use of the program laid out in source 4. It has a neutral relationship with source 3. It uses the definition provided by source 6.

      \item The use of this source would allow me to directly implement the voice of the founder of the Free Software Movement, Dr. Richard M. Stallman.

      \item Though missing a date and not peer-reviewed, the document is, first and foremost, not a technical research piece, but rather an argumentative piece. Thus, a peer review is not necessary. Furthermore, the piece is written by the founder of the movement himself, Dr. Richard M. Stallman. There is, undoubtedly, no one more credible to write about Free Software Issues than the founder of the movement.

    \end{itemize}

  \item ``What is Free Software?''

    \begin{itemize}

      \item Author: Free Software Foundation

      \item Journal: Free Software Foundation Archive

      \item Published: No date provided

      \item doi: \href{https://www.gnu.org/philosophy/free-sw.html}{https://www.gnu.org/philosophy/free-sw.html}

      \item This source provides an official definition for ``Free Software''

      \item No conclusion is necessary, as the piece simply provides a definition.

      \item All of the other sources rely on the definition provided within this source.

      \item I will use this source to provide a formal definition for Free Software, which will allow me to introduce to the problem.

      \item The use of this source would allow me to more easily identify the problem, and then provide solutions.

    \end{itemize}

\end{enumerate}

\end{document}

