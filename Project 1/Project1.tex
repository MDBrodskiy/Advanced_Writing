%%%%%%%%%%%%%%%%%%%%%%%%%%%%%%%%%%%%%%%%%%%%%%%%%%%%%%%%%%%%%%%%%%%%%%%%%%%%%%%%%%%%%%%%%%%%%%%%%%%%%%%%%%%%%%%%%%%%%%%%%%%%%%%%%%%%%%%%%%%%%%%%%%%%%%%%%%%%%%%%%%%
% Written By Michael Brodskiy
% Class: Advanced Writing in the Disciplines
% Professor: L. Nardone
%%%%%%%%%%%%%%%%%%%%%%%%%%%%%%%%%%%%%%%%%%%%%%%%%%%%%%%%%%%%%%%%%%%%%%%%%%%%%%%%%%%%%%%%%%%%%%%%%%%%%%%%%%%%%%%%%%%%%%%%%%%%%%%%%%%%%%%%%%%%%%%%%%%%%%%%%%%%%%%%%%%

\documentclass[12pt]{article} 
\usepackage{alphalph}
\usepackage[utf8]{inputenc}
\usepackage[russian,english]{babel}
\usepackage{titling}
\usepackage{amsmath}
\usepackage{graphicx}
\usepackage{enumitem}
\usepackage{amssymb}
\usepackage[super]{nth}
\usepackage{everysel}
\usepackage{ragged2e}
\usepackage{geometry}
\usepackage{multicol}
\usepackage{fancyhdr}
\usepackage{cancel}
\usepackage{siunitx}
\usepackage{setspace}
\usepackage[table]{xcolor}

\doublespacing

\geometry{top=1.0in,bottom=1.0in,left=1.0in,right=1.0in}
\newcommand{\subtitle}[1]{%
  \posttitle{%
    \par\end{center}
    \begin{center}\large#1\end{center}
    \vskip0.5em}%

}
\usepackage{hyperref}
\hypersetup{
colorlinks=true,
linkcolor=blue,
filecolor=magenta,      
urlcolor=blue,
citecolor=blue,
}

\usepackage{tcolorbox}

\pagestyle{fancy}

\lfoot[\vspace{-15pt} \hline]{\vspace{-15pt} \hline}
\rfoot[\vspace{-15pt} \hline]{\vspace{-15pt} \hline}
\cfoot[\thepage]{\thepage}
\chead[\textsc{Advanced Writing in the Disciplines}]{\textsc{Advanced Writing in the Disciplines}}
\lhead[\textsc{Project 1}]{\textsc{Project 1}}
\rhead[\textsc{ENGW3302}]{\textsc{ENGW3302}}



\begin{document}

\begin{center}

  \textbf{Project 1 — Disciplinary Introduction}

\end{center}

\begin{justify}

  \hspace{.5in} As many know, ``our whole universe was in a hot dense state, then nearly 14 billion years ago expansion started. Wait\ldots'' Physicists are all around us, and have directly contributed to many advancements (and catastrophes) since antiquity, ranging from the Greek fabrication of the concept of Earth in `space' to modern particle colliders, like those found at CERN. Currently, I am studying Electrical Engineering and Physics; however, for the sake of brevity, in tandem with greater personal interest, the scope of this exposition will limited to the latter. Furthermore, given my specialization and the multitude of concepts present within the field of physics, I will focus on some of my favorites (and fields I plan to further specialize in once I matriculate to a graduate program), namely electricity and magnetism, as well as high-energy particle and nuclear physics.

  \hspace{.5in} Electricity and magnetism, often shortened to electromagnetics due to the duality of the two, concerns, quite evidently, all phenomena pertaining to electricity and magnetism. This may include, but is not limited to: charges, forces, fields, currents, and much, much more. High-energy particle and particle physics are also two closely related fields, with the former tackling the nature and interactions between particles, and the latter exploring atomic nuclei. Thus far in my academic career, my coursework has covered these fields in a superficial manner, as it is difficult to understand the subtle, complex nature of these topics without delving into graduate level mathematics; however, despite this, I have already been involved with extensive laboratory work, which has resulted in the production of a variety of documentation

  \hspace{.5in} First and foremost, there is the sacred laboratory report. Such a report forms the crux of contemporary experimental physics, as it is a means to convey data produced from research endeavors. It is imperative for a physicist to closely follow documentation requirements (laid out by various organizations, depending on where and for whom research is being performed) when producing a laboratory report so as to allow for maximum efficiency when relaying information. Additionally, it is very common to find research physicists writing letters, as it is oftentimes necessary to apply for grants in order to fund research ventures. Ultimately, any document produced by a physicist is intended to reach the final product: a research paper. Research papers may be used to summarize data found in laboratory reports, as well as incorporate mathematical formulas and calculations, in order to reach a conclusion based on patterns observed. Such research papers, especially those of great importance and interest to the greater scientific community, are then peer-reviewed to verify the accuracy and veracity of the findings.

  \hspace{.5in} As such, peer-reviewed research papers can often be accessed via online repositories. One such online repository is ``Molecular Diversity Preservation International,'' shortened \href{https://www.mdpi.com/}{MDPI}. Given the aforementioned interdisciplinary areas of expertise present within physics, journal productions may range from biophysics to econophysics to nuclear physics. Such works, especially those of great significance, are often presented at conferences. Conferences are generally organized on the international scale, by organizations like the International Union of Pure and Applied Physics (\href{https://iupap.org/}{IUPAP}).

  \hspace{.5in} Despite the inundation of works being produced daily, there are many questions we still need to tackle before we discover everything the universe has to offer. Perhaps the most perplexing and persistent of such questions is whether or not it is possible to delineate a unified field theory. Einstein dedicated his whole life to this effort, and was successful \ldots at least for a time. Previously, only two fields were known: electromagnetic and gravitational. Through the discovery of relativity, Einstein was able to correlate the two fields; however, during his research, two new fields — the strong and weak nuclear forces — would be discovered. This occurred as a result of new inquiries into particle and nuclear physics. Though the concept of gravity is fairly familiar to most people, and electromagnetism is still fairly familiar, albeit not most, the discovery of these two new fields was a shocking development for the physics community, especially at the time. The weak nuclear force holds the electrons, which surround an atom's nucleus (consisting of protons and neutrons) in `shells' and `orbitals.' This force governs interactions between atoms, and is used to explicate nuclear interactions, or, more specifically, fusion and fission. The weak nuclear force is approximately a million times stronger than gravitation, which explains its applicability to energy production in the form of nuclear reactors. On the other hand, the strong nuclear force holds together an atom's nucleus. It is more than 100 times one undecillion (or a million times itself 6 times, that's 38 zeros!) stronger than gravity. This force drives the ideals of nuclear armaments, and is only \textit{partially} (not fully) released upon nuclear bomb detonation (at this point, I think most, if not all, of us have seen Oppenheimer).

  \hspace{.5in} Although many simply know one-off, singular names, like Albert Einstein, J. Robert Oppenheimer, or Neil deGrasse Tyson, physics is one of the most, if not \textit{the} most, collaborative and community-driven fields there is. As a matter of fact, in 2015, a physics publication, related to work at the large hadron collider at CERN, broke a record for the most contributors, with \href{https://www.nature.com/articles/nature.2015.17567}{5,154 physicists} working on the project. This is a record not only within the physics community, but for any research article ever produced. \href{https://arxiv.org/pdf/1901.02789}{A census} of physics-related related publications produced between 1985 and 2015 identified over 135,877 physicists making contributions within these years — one of the largest scientific research bodies. Personally, I have produced \href{https://arxiv.org/abs/2405.01727}{a publication} (in collaboration with another physicist), which recently got approved to be added to \href{https://arxiv.org/}{arXiv}, a major publication repository (albeit not peer-reviewed). This work examines a field known as mathematical physics, more specifically the applicability of certain mathematical operations to computer science (large language models, or LLMs). In any case, collaboration is a crucial skill for physicists to have, as it is ubiquitous within research.

  \hspace{.5in} That being said, according to the \href{https://www.aps.org/}{American Physical Society}, physics seems to be a predominantly white-dominated field. According to the graphic found \href{https://www.aps.org/programs/education/statistics/degreesbyrace.cfm}{here}, approximately 70\% of physics degree holders are white, with 13.2\% being Asian, a stark discrepancy between. Additionally, the \href{https://www.aip.org/}{American Institute of Physics} reports that only 20\% of degrees granted in 2017 were women. Hopefully, with work, this strong correlation between white males and physics degrees may be lessened, which would allow the field to benefit from greater equity and more diverse perspectives.

  \hspace{.5in} Overall, I hope to personally contribute to the inquiries into the world around us, and the ever-present phenomena that govern it. I am working towards constructing my applications to graduate programs, which will hopefully secure my candidacy for a doctorate in either nuclear or particle physics. My existing experience and knowledge in the fields of mathematics, physics, and computer science (and even my knowledge of multiple languages) would allow me to be a qualified, effective, and even prolific member of the scientific community. Furthermore, using my multilingual and multicultural background, I hope to not only contribute to the body of knowledge of physics, but also to bridge gaps that divide us across national boundaries, ultimately creating a robust, global scientific community.

\end{justify} 

\end{document}

