%%%%%%%%%%%%%%%%%%%%%%%%%%%%%%%%%%%%%%%%%%%%%%%%%%%%%%%%%%%%%%%%%%%%%%%%%%%%%%%%%%%%%%%%%%%%%%%%%%%%%%%%%%%%%%%%%%%%%%%%%%%%%%%%%%%%%%%%%%%%%%%%%%%%%%%%%%%%%%%%%%%
% Written By Michael Brodskiy
% Class: Advanced Writing in the Disciplines
% Professor: L. Nardone
%%%%%%%%%%%%%%%%%%%%%%%%%%%%%%%%%%%%%%%%%%%%%%%%%%%%%%%%%%%%%%%%%%%%%%%%%%%%%%%%%%%%%%%%%%%%%%%%%%%%%%%%%%%%%%%%%%%%%%%%%%%%%%%%%%%%%%%%%%%%%%%%%%%%%%%%%%%%%%%%%%%

\documentclass[12pt]{article} 
\usepackage{alphalph}
\usepackage[utf8]{inputenc}
\usepackage[russian,english]{babel}
\usepackage{titling}
\usepackage{amsmath}
\usepackage{graphicx}
\usepackage{enumitem}
\usepackage{amssymb}
\usepackage[super]{nth}
\usepackage{everysel}
\usepackage{ragged2e}
\usepackage{geometry}
\usepackage{multicol}
\usepackage{fancyhdr}
\usepackage{cancel}
\usepackage{siunitx}
\usepackage{setspace}
\usepackage[table]{xcolor}

\doublespacing

\geometry{top=1.0in,bottom=1.0in,left=1.0in,right=1.0in}
\newcommand{\subtitle}[1]{%
  \posttitle{%
    \par\end{center}
    \begin{center}\large#1\end{center}
    \vskip0.5em}%

}
\usepackage{hyperref}
\hypersetup{
colorlinks=true,
linkcolor=blue,
filecolor=magenta,      
urlcolor=blue,
citecolor=blue,
}

\usepackage{tcolorbox}

\pagestyle{fancy}

\lfoot[\vspace{-15pt} \hline]{\vspace{-15pt} \hline}
\rfoot[\vspace{-15pt} \hline]{\vspace{-15pt} \hline}
\cfoot[\thepage]{\thepage}
\chead[\textsc{Advanced Writing in the Disciplines}]{\textsc{Advanced Writing in the Disciplines}}
\lhead[\textsc{Project 1}]{\textsc{Project 1}}
\rhead[\textsc{ENGW3302}]{\textsc{ENGW3302}}



\begin{document}

\begin{tcolorbox}
\begin{center}

  \textbf{\underline{Author's Note}}

\end{center}

\begin{justify}

  \setstretch{1.4}

  \vspace{-15pt}

  \hspace{.5in} Fabrication of this work began with a personal reflection regarding which aspects of physics I wanted to focus on. Though I received comments to widen the scope of this document to embody the ideals of overall physics, rather than my favorite aspects, I decided against this. The reason for this was two-fold: first, there exists no such career path as simply 'physicist' — that is, a more precise specialization, such as 'nuclear,' 'environment,' or 'condensed matter' always accompanies the title — and, second, such a document is simply too compact to house enough information to appropriately describe all of the possible career paths that exist for a physicist. Given this, I concluded that the best course of action would be to focus on my personally planned career path (fueled by interest since watching HBO's Chernobyl) in the physics of energy generation (a combination of nuclear and electromagnetic principles).
  
  \hspace{.5in} Albeit set in the focus of the paper, I had to decide where to flow from the introduction. Given that physics-related work is divided into two main branches — experimental and theoretical — I once again decided to focus on my personal greater interest (and also that which is more applicable to both nuclear and electromagnetic physics): experimental. Upon making this decision, I wanted to get more information on what kind of scientific documentation exists beyond my own knowledge (there was not much to be found, though). I did, however, come across an article detailing the collaborative effort involving thousands of physicists that came to be known as the scientific publication with the greatest amount of contributors ever (and knew this would be a perfect addition to the paper).

  \hspace{.5in} Most importantly, I received a comment inquiring about my own (recent) publication, as well as another comment about my cultural background. Upon reading this, I thought that it was insufficient to simply mention my publication, and, furthermore, I could add some of my own voice to the document by including more information about my personal experience. Despite this, I decided against discussing my own cultural background to eschew detracting from a formal tone.

  \hspace{.5in} As I implemented the above changes, in addition to developing my analysis, I thought that a final proof-read was necessary, and thought it necessary to modify a few words here and there to improve the flow of the overall work. Hope you enjoy!

\end{justify}

\end{tcolorbox}

\begin{center}

  \textbf{\underline{Disciplinary Introduction}}

\end{center}

\begin{justify}

  \hspace{.5in} As many know, ``our whole universe was in a hot dense state, then nearly 14 billion years ago expansion started. Wait\ldots'' Physicists are all around us, and have directly contributed to many advancements (and catastrophes) since antiquity, ranging from the Greek fabrication of the concept of Earth in `space' to modern particle colliders, like those found at CERN. Given my specialization in electrical engineering, as well as the multitude of concepts present within the field of physics, I will focus on some of my personal favorites (and fields I plan to further specialize in once I matriculate to a graduate program), namely electricity and magnetism, as well as high-energy particle and nuclear physics.

  \hspace{.5in} Electricity and magnetism, shortened to electromagnetics, concerns, all phenomena pertaining to electricity and magnetism. This may include: charges, forces, fields, and currents. High-energy particle and nuclear physics are also two closely related fields, with the former tackling interactions between particles, and the latter exploring atomic nuclei. Physicists in such fields could be working on, for example, energy production in the form of a fusion reactor, which would utilize heat generated from fusion of two particles to create electricity (currently under development). Though this intrigues me, thus far in my academic career, my coursework has covered these fields in only a superficial manner, as it is difficult to quantitatively define the complex nature of these topics without delving into graduate level mathematics; however, despite this, I have already been involved with extensive laboratory work, which has resulted in the production of a variety of documentation. Finding credible and interesting laboratory work may be of great difficulty for young physicists, and should be pursued as early as possible to create publications, which improve one's stance within the community.

  \hspace{.5in} There are many ways to contribute, and, first and foremost,  is the sacred laboratory report. Such a report forms the crux of contemporary experimental physics, as it conveys data produced from research. Physicists must follow documentation requirements (laid out by various organizations, depending on where and for whom research is being performed) when producing a laboratory report so as to allow for maximum efficiency when relaying information. Additionally, research physicists often write letters to apply for grants. Ultimately, any document produced by a physicist is intended to reach the final product: a research paper. Research papers may summarize data found in laboratory reports, as well as incorporate mathematical formulas and calculations to reach a conclusion based on patterns observed. Such research papers, especially those of great importance and interest to the greater scientific community, are then peer-reviewed to verify the accuracy and veracity of the findings. This demonstrates the collective nature of physics (and scientific work in general).

  \hspace{.5in} Peer-reviewed research papers can often be accessed via online repositories, like ``Molecular Diversity Preservation International,'' shortened \href{https://www.mdpi.com/}{MDPI}. Because of interdisciplinary expertise, journal topics may range from biophysics to econophysics to nuclear physics. Such works, especially those of great significance (as determined by community interest), are often presented at conferences. Conferences are generally organized on the international scale, by organizations like the International Union of Pure and Applied Physics (\href{https://iupap.org/}{IUPAP}). Given the large-scale nature of such efforts, it is often difficult for new physicists to come to light. Since only notable research efforts receive attention, and young physicists (especially without graduate degrees) are rarely invited (or qualified) to participate in such endeavors, it has become crucial, now more than ever, to start developing publications and a general sense of research topics as early as possible.

  \hspace{.5in} Despite the large quantity of existing and ongoing publications, the universe still has much to offer. Perhaps the most perplexing question is whether or not it is possible to delineate a unified field theory. Einstein dedicated his whole life to this effort, and was successful \ldots for a time. Previously, only two fields were known: electromagnetic and gravitational. Through the discovery of relativity, Einstein correlated the two fields; however, during his endeavor, two new fields — the strong and weak nuclear forces — were discovered. This occurred as a result of new inquiries into particle physics. Though the concept of gravity is fairly familiar to most people, and electromagnetism is still familiar, albeit to a lesser few, the discovery of these two new fields was a shocking development for the community. The weak nuclear force holds electrons, which surround an atom's nucleus (consisting of protons and neutrons) in `orbitals.' This force governs interactions between atoms, and is used to explicate fusion and fission. The weak nuclear force is approximately a million times stronger than gravitation, which explains its applicability to energy production in the form of nuclear reactors. On the other hand, the strong nuclear force holds together an atom's nucleus. It is more than 100 undecillion (a million times itself 6 times, or 38 zeros!) stronger than gravity. This force drives the ideals of nuclear armaments, and is only \textit{partially} released upon nuclear bomb detonation. This is quite difficult to imagine; however, most importantly, it demonstrates both an impressive feat and unimaginable catastrophe, as it made the annihilation of civilization less of an idea and more of a possibility, especially given man's tendency to war.

  \hspace{.5in} Although many simply know one-off, singular names, like Albert Einstein, J. Robert Oppenheimer, or Neil deGrasse Tyson, physics is one of the most, if not \textit{the} most, collaborative and community-driven fields there is. As a matter of fact, in 2015, a physics publication, related to work at the large hadron collider at CERN, broke a record for the most contributors, with \href{https://www.nature.com/articles/nature.2015.17567}{5,154 physicists} working on the project. This is a record not only within the physics community, but for any research article ever produced. \href{https://arxiv.org/pdf/1901.02789}{A census} of physics-related publications produced between 1985 and 2015 identified over 135,877 physicists making contributions within these years — one of the largest scientific research bodies. Personally, I have produced \href{https://arxiv.org/abs/2405.01727}{a publication} (in collaboration with another physicist), which recently got approved to be added to \href{https://arxiv.org/}{arXiv}, a major publication repository (albeit not peer-reviewed). This work examines a field known as mathematical physics, more specifically the applicability of certain mathematical operations to computer science (large language models, or LLMs). My own experience with collaboration allowed me to understand what the field is truly about: discussion. No work, no matter how small, would be fully developed within physics if it weren't for open discussions. Imagine, for example, the design of the hydrogen bomb, as shown in the aforementioned Oppenheimer movie. Oppenheimer was intelligent enough to understand the governing principles behind atomic fission; however, he would never have been able to fully flesh out his ideas without an open discussion, which allowed flaws to come to light, and robust ideas to be further developed. Such open-forum discussion is applicable in any given situation, whether it is my publication (involving discussion between myself and my peer), or the publication with several thousand contributors — cognitive bias will exist in any scenario. In any case, collaboration is a crucial skill for physicists to have, as it is ubiquitous within research. This means that young physicists, if interested in pursuing research, must hone their skills when it comes to collaboration. This may include learning to use \LaTeX (LaTeX — which was used to produce this document as well!), a common document creation tool in the Physics community, being able to convey ideas effectively by simply talking it out, and learning to defend one's stance (which is especially important when pursuing a doctoral degree). As such, it is further evident that becoming known within the physics community may be quite difficult.

  \hspace{.5in} That being said, there are also uncontrollable barriers to entry for new physicists. According to the \href{https://www.aps.org/}{American Physical Society}, physics seems to be a predominantly white-dominated field. According to the graphic found \href{https://www.aps.org/programs/education/statistics/degreesbyrace.cfm}{here}, approximately 70\% of physics degree holders are white, with 13.2\% being Asian, a stark discrepancy between. Additionally, the \href{https://www.aip.org/}{American Institute of Physics} reports that only 20\% of degrees granted in 2017 were women. Hopefully, with work, this strong correlation between white males and physics degrees may be lessened, which would allow the field to benefit from greater equity and more diverse perspectives.

  \hspace{.5in} Overall, I hope to personally contribute to the inquiries into the world around us, and the ever-present phenomena that govern it. I am working towards constructing my applications to graduate programs, which will hopefully secure my candidacy for a doctorate in either nuclear or particle physics. My existing experience and knowledge in the fields of mathematics, physics, and computer science (and even my knowledge of multiple languages) would allow me to be a qualified, effective, and even prolific member of the scientific community. Furthermore, using my multilingual and multicultural background, I hope to not only contribute to the body of knowledge of physics, but also to bridge gaps that divide us across national boundaries, ultimately creating a robust, global scientific community.

\end{justify} 

\begin{center}
  \begin{tcolorbox}[colback=blue!5,colframe=blue!15,width=4in]
    \begin{center}
    \begin{tabular}{|c|c|}
      \hline
      \cellcolor{black} \textcolor{white}{\textbf{Section}} & \cellcolor{black} \textcolor{white}{\textbf{Word Count}}\\
      \hline
      \rowcolor{white} Author's Note & 383\\
      \hline
      \rowcolor{white} Disciplinary Introduction & 1,491\\
      \hline
    \end{tabular}
    \end{center}
  \end{tcolorbox}
\end{center}

\end{document}

