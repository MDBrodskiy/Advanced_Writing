%%%%%%%%%%%%%%%%%%%%%%%%%%%%%%%%%%%%%%%%%
% Memo
% LaTeX Template
% Version 1.0 (30/12/13)
%
% This template has been downloaded from:
% http://www.LaTeXTemplates.com
%
% Original author:
% Rob Oakes (http://www.oak-tree.us) with modifications by:
% Vel (vel@latextemplates.com)
%
% License:
% CC BY-NC-SA 3.0 (http://creativecommons.org/licenses/by-nc-sa/3.0/)
%
%%%%%%%%%%%%%%%%%%%%%%%%%%%%%%%%%%%%%%%%%

\documentclass[letterpaper,11pt]{texMemo} % Set the paper size (letterpaper, a4paper, etc) and font size (10pt, 11pt or 12pt)

\usepackage{parskip} % Adds spacing between paragraphs
\setlength{\parindent}{15pt} % Indent paragraphs
\usepackage{censor}

\usepackage{hyperref}
\hypersetup{
colorlinks=true,
linkcolor=blue,
filecolor=magenta,      
urlcolor=blue,
citecolor=blue,
}

%----------------------------------------------------------------------------------------
%	MEMO INFORMATION
%----------------------------------------------------------------------------------------

\memoto{\blackout{Xxxxxxxx Xxxxxxxx and Xxxxxx Xxxxxxx}} % Recipient(s)

\memofrom{Michael Brodskiy} % Sender(s)

\memosubject{Informative Article Review Memo} % Memo subject

\memodate{Tuesday, June 11, 2024} % Date, set to \today for automatically printing todays date

\logo{\includegraphics[width=0.3\textwidth]{logo.png}} % Institution logo at the top right of the memo, comment out this line for no logo

%----------------------------------------------------------------------------------------

\begin{document}

\maketitle % Print the memo header information

%----------------------------------------------------------------------------------------
%	MEMO CONTENT
%----------------------------------------------------------------------------------------

\blackout{Xxxxxxxx},

This piece was quite effective at conveying both the magnitude and urgency of the situation. I particularly enjoyed how you formatted it to have a large, eye-catching title, with a smaller sub-heading, and finally the actual body.

The work begins by identifying prominent and pertinent software and hardware, like ChatGPT and Apple. This is used to flow into the topic of AI use itself, the urgency of which is emphasized by comparing energy usage to the country of Ireland. Some alternatives, such as more efficient programming or specialized hardware are provided; however, these solutions are very high-level, and hard to directly implement. As such, from here, rather than further focusing on the electrical usage, the piece focuses on a byproduct of the usage: heat generation. Instead of simply stating a solution like ``use less heat,'' a more effective solution is provided: funnel the generated heat into cities using a heat pump. This heat can then be used an energy efficient alternative to conventional heating methods. The article concludes by enunciating the consumer's role in the energy expenditure, and urges them to consider if an artificial intelligence search is actually necessary.

In terms of high level improvements, there really is not too much that needs to be adjusted. All that should be considered overall is the possibility of lengthening the piece itself, as it seems a bit rushed. This could be done by introducing some more specific details and/or statistics, possibly through a medium such as an infographic. In doing so, the piece would feel a bit more ``whole.''

In terms of smaller fixes, there appears to be a tendency for run-on sentences, similar to those found in Project 2. Additionally, some of the sentences seem to be incomplete. For example, consider modifying the following sentences to ensure they are complete: ``There are many ways to do this by buying specialized hardware or coding more efficiently\underline{. But} companies still often resort to energy inefficient methods simply out of convenience.''

Overall, keep up the good work!

Respectfully,
Michael Brodskiy


%----------------------------------------------------------------------------------------

\end{document}
